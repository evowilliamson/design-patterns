\documentclass[a4paper]{article}

\usepackage{xcolor}
\usepackage{fancyheadings}


\newcommand{\todo}[1]{\textcolor{red}{[#1]}}
\lhead{Open Universiteit}
\chead{IM0102, Design patterns}
\rhead{Eindopdracht}

\begin{document}
\pagestyle{fancy}

\section*{Studentgegevens}
\begin{description}
	\item [Cursuscode] IM0102
	\item Interactieve Navigatie
	\item [Naam]Randy Pottgens
	\item [Studentnummer]851941098
	\item [Naam]Ivo Willemsen
	\item [Studentnummer]851926289
\end{description}

\section*{Aanpak}
\todo{<Geef aan hoe jullie de opdracht hebben aangepakt en wie wat heeft gedaan, maximaal 1 A-4. Geef expliciet aandacht aan de volgorde van activiteiten>}



\section{Problem analysis}
Jabberpoint is a simple slide show application that can read a slide show from a source, allows the user to navigate through the slides and save the \textit{state} of the running slide show to the source again.
\\*The main concept is the \textbf{slide show}. A slide show consists of the following parts:
\begin{itemize}
\item A \textbf{head}, which consists of a \textbf{title} and a possible \textbf{theme} (ask teacher)
\item A list of \textbf{slides}. There must be at least one slide present in the slide show. Slides in a slide show have a predefined order
\item A list of \textbf{threads} (ask teacher). 
\end{itemize}
The application does not provide the possibility of editing the slides within the slide show


\section{Ontwerp}


\section{Keuzen}



\section{Sourcecode}

\end{document}
